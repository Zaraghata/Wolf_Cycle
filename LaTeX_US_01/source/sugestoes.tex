\chapter[Sugestões]{Sugestões}
\label{chap:sugestoes}
	\section[Encenação]{Encenação}
	\label{sec:sugestoes_encenacao}
		A atividade central desse passo envolve apresentar o capítulo \textbf{Os Irmãos de Mowgli} do \textbf{Livro da Selva}, de Rudyard Kipling (ver Seção \ref{sec:referencias_livroDaSelva}), aos novos lobinhos que iniciarão sua caminhada no movimento escoteiro.
		\\ \indent Para tal fim, a atividade de encenação é uma ótima opção para que os Velhos Lobos apresentem a história à Alcateia. Há diversas maneiras de se encenar, descrevemos ao longo deste passo alguns desses tipos de encenação, são eles: \nameref{subsec:sugestoes_encenacao_teatro}, \nameref{subsec:sugestoes_encenacao_curtaMetragem}, \nameref{subsec:sugestoes_encenacao_historiaContada} e \nameref{subsec:sugestoes_encenacao_trilhaEncenada}.
		\\ \indent Toda atividade de encenação envolve a construção de um roteiro, composto por cenas que serão postos em prática pelo(s) ator(es) em um cenário específico.
		\\ \indent Divida e desenvolva as cenas relevantes de acordo com a história de forma que contextualize os pontos importantes da narrativa. Lembrem-se de direcionar o roteiro ao seu público-alvo, os lobinhos, que devem se sentir como se estivessem vivendo as aventuras de Mowgli.
		\\ \indent Os personagens deverão ser descritos como papéis e possuírem um figurino próprio, definidos de acordo com a personalidade dos Velhos Lobos da Alcateia, de forma que, cada irmão de Mowgli seja interpretado pelo Velho Lobo correspondente. O objetivo é que o lobinho entenda que aquele personagem estará o acompanhando durante todo seu caminho no Ramo Lobo.

		\subsection[Teatro]{Teatro}
		\label{subsec:sugestoes_encenacao_teatro}
			\{ Responsável para Desenvolver: Henrique Campos \}

		\subsection[Curta-Metragem]{Curta-Metragem}
		\label{subsec:sugestoes_encenacao_curtaMetragem}
			A opção de \nameref{subsec:sugestoes_encenacao_curtaMetragem} é recomendavel à Alcateia cujos Velhos Lobos queiram realizar a encenação antes do definitivo primeiro contato com os novos lobinhos, evitando eventuais problemas que podem ocorrer em uma encenação ao vivo. De pontos positivos em se desenvolver um \nameref{subsec:sugestoes_encenacao_curtaMetragem}, podemos destacar:
			\begin{itemize}
				\item{De forma geral, um \nameref{subsec:sugestoes_encenacao_curtaMetragem} possui um grande poder de ambientação e imersão ao espectador;}
				\item{É possível reutilizar um \nameref{subsec:sugestoes_encenacao_curtaMetragem};}
				\item{Evita o improviso, sendo capaz de atingir o resultado esperado sem a presença do público.}
			\end{itemize}
			\ \indent De pontos negativos, podemos destacar:
			\begin{itemize}
				\item{De forma geral, um \nameref{subsec:sugestoes_encenacao_curtaMetragem} consome maior tempo de desenvolvimento;}
				\item{É necessário aprender conceitos de filmagem e edição, além dos tradicionais conceitos de encenação;}
				\item{Se houver mudanças nos Velhos Lobos, como a saída ou a entrada de um novo integrante, se faz necessário a construção de um novo \nameref{subsec:sugestoes_encenacao_curtaMetragem}.}
			\end{itemize}
			\begin{itemize}
				\item[]{\large{\textbf{Tarefas}}}
				\item[\LARGE{$\square$}]{\textbf{1.} Desenvolver um Roteiro;}
				\item[\LARGE{$\square$}]{\textbf{2.} Relacionar Custos de Produção;}
				\item[\LARGE{$\square$}]{\textbf{3.} Organizar Equipe de Produção e Atuação;}
				\item[\LARGE{$\square$}]{\textbf{4.} Definir Cronograma de Filmagem;}
				\item[\LARGE{$\square$}]{\textbf{5.} Realizar as Filmagens das Cenas;}
				\item[\LARGE{$\square$}]{\textbf{6.} Realizar a Edição das Cenas;}
			\end{itemize}
			\ \indent Antes do processo de produção do curta, é recomendável ler o livro \textbf{O Manual do Cineasta} de Steven Ascher e Edward Pincus, livro que visa integrar o leitor aos processos de produção cinematográfica. Recomenda-se ler também \textbf{Os Doze Passos do Herói} de Joseph Campbell para melhor entendimento de como interpretar os personagens.
			\\ \indent Após a contextualização é de sumo importância buscar a ambientação para gravar as cenas do curta, de forma que seja filmado em um local comum para os lobinhos, de preferência no local predominante de desempenho das futuras atividades dos lovinhos.
			\\ \indent Na realização das filmagens, deverão ser utilizados técnicas de corte e tomadas de cenas. Tente seguir essas características:
			\begin{itemize}
				\item{\emph{\textbf{Long Shot}} (ou filmagem à longa distância, plano geral): Usado para ambientação, apresentação do cenário, filmagens de ação ampla, usada como captura de plano geral;}
				\item{\emph{\textbf{Medium Shot}}(ou filmagem à média distância, plano simples): Usado principalmente para mostrar as pessoas que estão presentes na cena, de forma que caputre a linguagem corporal, sem grande destaque para expressões faciais. Usado também para filmagens de diálogos;}
				\item{\emph{\textbf{Close-Up}}(ou filmagem de evidenciação, plano fechado): Usado para dar foco em algo na cena. Expressões faciais, objetos-chave, entre outras situações que demandem maior ênfase.}
			\end{itemize}
			\{ Para fazer: Atuação (Wesley) \}
			\\ \indent A edição das filmagens é uma importante tarefa de desenvolvimento de um \nameref{subsec:sugestoes_encenacao_curtaMetragem}, nela serão integrados os efeitos audiovisuais às cenas filmadas anteriormente utilizando de um programa de edição de vídeos. Busque integrar as cenas de forma que dê uma perspectiva de continuidade da história, tome liberdade para propor novas filmagens se necessário.

		\subsection[História Contada]{História Contada}
		\label{subsec:sugestoes_encenacao_historiaContada}
			\{ Responsável para Desenvolver: Wesley \}

		\subsection[Trilha Encenada]{Trilha Encenada}
		\label{subsec:sugestoes_encenacao_trilhaEncenada}
			A opção de \nameref{subsec:sugestoes_encenacao_trilhaEncenada} é recomendável à Alcateia que deseja realizar o passo em um ambiente natural, como um parque ecológico. As características da \nameref{subsec:sugestoes_encenacao_trilhaEncenada} são semelhantes ao \nameref{subsec:sugestoes_encenacao_teatro}, com a diferença de que a encenação ocorrerá em uma trilha, onde a história e seus personagens serão apresentados à medida que o público caminha nessa trilha. É uma ótima opção se for possível caracterizar o cenário da história ao ambiente.
			\\ \indent De pontos positivos em se desenvolver uma \nameref{subsec:sugestoes_encenacao_trilhaEncenada}, podemos destacar:
			\begin{itemize}
				\item{De forma geral, um \nameref{subsec:sugestoes_encenacao_trilhaEncenada} possui um grande poder de ambientação e imersão ao espectador;}
				\item{Os lobinhos, como público, terá um papel ativo na encenação, participando da história à medida que percorre a trilha;}
			\end{itemize}
			\ \indent De pontos negativos, podemos destacar:
			\begin{itemize}
				\item{Requer improvisação e boa atuação para os personagens;}
				\item{É limitado à um ambiente sensível à história original;}
				\item{É sensível às condições climáticas, por ser realizado em um ambiente aberto;}
			\end{itemize}
			\begin{itemize}
				\item[]{\large{\textbf{Tarefas}}}
				\item[\LARGE{$\square$}]{\textbf{1.} Desenvolver um Roteiro;}
				\item[\LARGE{$\square$}]{\textbf{2.} Relacionar Custos de Produção;}
				\item[\LARGE{$\square$}]{\textbf{3.} Organizar Equipe de Produção e Atuação;}
				\item[\LARGE{$\square$}]{\textbf{4.} Definir Cronograma de Ensaios e Produção;}
				\item[\LARGE{$\square$}]{\textbf{5.} Realizar Ensaios;}
				\item[\LARGE{$\square$}]{\textbf{6.} Apresentar Trilha Encenada;}
			\end{itemize}
			\ \indent Durante o desenvolvimento do roteiro, proponha momentos em que os lobinhos possam interagir com a cena e com a história, principalmente quando houver as entre-cenas, momento de transição entre a primeira cena e a cena seguinte, onde ocorrerá a caminhada na trilha.
			\\ \indent Também é importante desenvolver um segundo roteiro, muito semelhante ao original para um ambiente fechado, caso ocorra condições climáticas desfavoráveis no dia.