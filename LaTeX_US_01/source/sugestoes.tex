\chapter[Sugestões]{Sugestões}
\label{chap:sugestoes}
	\section[Encenação]{Encenação}
	\label{sec:sugestoes_encenacao}
		Seguindo a história, divida as cenas relevantes, de forma que contextualize o necessário para que os lobinhos se sintam integrados à jângal. Leve em consideração que os lobinhos devem se sentir no papel do Mowgli. Para tal deverá ser trabalhado técnicas de inclusão do público à peça.
		\\ \indent Papéis deverão ser definidos de acordo com a personalidade dos velhos lobos, de forma que, cada irmão de Mowgli seja interpretado pelo velho lobo relativo, pois o intuito é que o lobinho entenda que aquele personagem estará o acompanhando durante seu processo do Integrar.

		\subsection[Teatro]{Teatro}
		\label{subsec:sugestoes_encenacao_teatro}
			\{ Responsável para Desenvolver: Henrique Campos \}

		\subsection[Curta-Metragem]{Curta-Metragem}
		\label{subsec:sugestoes_encenacao_curtaMetragem}
			A opção de \nameref{subsec:sugestoes_encenacao_curtaMetragem} cabe bem às necessidades de uma matilha cujos Velhos Lobos queiram poupar tempo de ensaios e trabalhos prévios, visto que a gravação poderá ser feita apenas uma vez e depois apresentado aos lobinhos. Porém, os chefes lobos deverão desempenhar um trabalho de qualidade e estar atualizando o curta de acordo com a necessidade da troca dos personagens.
			\\ \indent Antes do processo de produção do curta, é recomendável ler o livro \textbf{O Manual do Cienasta} de Steven Ascher e Edward Pincus, livro que visa integrar o leitor aos processos de produção cinematográfica. Recomenda-se ler também \textbf{Os Doze Passos do Herói} de Joseph Campbell para melhor entendimento de como interpretar os personagens.
			\\ \indent Após a contextualização é de sumo importância buscar a ambientação para gravar as cenas do curta, de forma que seja filmado em um local comum para os lobinhos, de preferência no local predominante de desempenho das futuras atividades dos lovinhos.
			\\ \indent Na realização das filmagens, deversão ser utilizados técnicas de angulação de câmera e proporção de cena. Tente seguir as seguintes instruções:
			\begin{itemize}
				\item{\emph{\textbf{Long Shot}}(ou filmagem à longa distância): Usado para ambientação, apresentação do cenário, filmagens de ação ampla, usada como captura de plano geral;}
				\item{\emph{\textbf{Medium Shot}}(ou filmagem à média distância): Usado principalmente para mostrar as pessoas que estão presentes na cena, de forma que caputre a linguagem corporal, sem grande destaque para expressões faciais. Usado também para filmagens de diálogos;}
				\item{\emph{\textbf{Close-Up}}(ou filmagem de evidenciação): Usado para dar foco em algo na cena. Expressões faciais, objetos-chave, entre outras situações que demandem maior ênfase.}
			\end{itemize}
			\{ Para fazer: Atuação (Wesley) \}
			\{ Para fazer: Edição (Jonathan) \}

		\subsection[História Contada]{História Contada}
		\label{subsec:sugestoes_encenacao_historiaContada}
			\{ Responsável para Desenvolver: Wesley \}

		\subsection[Trilha Encenada]{Trilha Encenada}
		\label{subsec:sugestoes_encenacao_trilhaEncenada}
			\{ Responsável para Desenvolver: Jonathan \}